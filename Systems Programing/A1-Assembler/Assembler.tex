\documentclass[12pt]{article}
\usepackage{amsmath}
\usepackage{listings}
\usepackage{color}
\usepackage{geometry}
\usepackage{float} % Required for the [H] float option
\usepackage{graphicx} % Required for including images
\geometry{margin=1in}
\usepackage{caption}
\usepackage[colorlinks=true, linkcolor=black, urlcolor=blue, citecolor=blue]{hyperref}
\usepackage{fancyhdr}
\usepackage{color}
\usepackage{xcolor} % Needed for custom colors
\definecolor{purple}{rgb}{0.5,0.0,0.5} % Define purple manually
\definecolor{codegray}{rgb}{0.5,0.5,0.5}
\definecolor{backcolour}{rgb}{0.95,0.95,0.92}


\pagestyle{fancy}
\fancyhead[L]{ Assignment -1 }
\fancyhead[R]{Gandholi Sarat - 23008}
\title{\textbf{PMAT 402 - Systems Programming \\ Assignment -1 \\ Two-Pass Assembler}}
\author{\textbf{Gandholi Sarat - 23008}}
\author{Gandholi Sarat - 23008}
\date{April 11, 2025}

\definecolor{codegray}{rgb}{0.5,0.5,0.5}
\definecolor{backcolour}{rgb}{0.95,0.95,0.92}
\lstdefinestyle{mystyle}{
    backgroundcolor=\color{backcolour},   
    commentstyle=\color{codegray},
    keywordstyle=\color{blue},
    numberstyle=\tiny\color{codegray},
    stringstyle=\color{purple},
    basicstyle=\ttfamily\footnotesize,
    breakatwhitespace=false,         
    breaklines=true,                 
    captionpos=b,                    
    keepspaces=true,                 
    numbers=left,                    
    numbersep=5pt,                  
    showspaces=false,                
    showstringspaces=false,
    showtabs=false,                  
    tabsize=2
}
\lstset{style=mystyle}

\begin{document}

\maketitle

\tableofcontents
\newpage

\section{Objective}
The objective of this project is to design and implement a Two-Pass Assembler for the Simplified Instructional Computer (SIC) architecture. The assembler translates assembly language programs into object code, manages a symbol table, and handles all standard SIC instructions and directives.

\section{Idea of Implementation}
The assembler works in two passes:

\begin{itemize}
    \item \textbf{Pass 1:} 
    \begin{itemize}
        \item Initializes location counter.
        \item Parses each line to extract label, opcode, operand.
        \item Builds the symbol table (SymTab) with label-address mappings.
        \item Generates an intermediate file with line addresses.
    \end{itemize}
    
    \item \textbf{Pass 2:}
    \begin{itemize}
        \item Reads the intermediate file and symbol table.
        \item Uses the operation table (OpTab) to fetch opcode for instructions.
        \item Replaces symbols with actual addresses.
        \item Generates object code records (Header, Text, End).
    \end{itemize}
\end{itemize}

\section{Functions Description}
Below are the key functions used in the assembler, along with their purpose and code implementation.

\subsection{Hexadecimal to Decimal Conversion}
This function converts a hexadecimal string (e.g., \texttt{"1A3"}) into its equivalent decimal integer. It processes each digit from right to left, multiplying it by powers of 16 based on its position.

\begin{lstlisting}[language=C++, caption={hextodec() - Hex to Decimal}]
int hextodec(string hex) {
    int len = hex.length();
    int base = 1;
    int dec_num = 0;
    for (int i = len - 1; i >= 0; i--) {
        dec_num += dec(hex[i]) * base;
        base *= 16;
    }
    return dec_num;
}
\end{lstlisting}

\subsection{Decimal to Hexadecimal Conversion}
This function takes a decimal integer and converts it into a hexadecimal string. It uses a stack to reverse the remainders when dividing the number by 16, constructing the hexadecimal result from most to least significant digit.

\begin{lstlisting}[language=C++, caption={dectohex() - Decimal to Hex}]
string dectohex(int dec_num) {
    string hex_str = "";
    stack<int> stack;
    int div = dec_num;
    int rem = div % 16;
    stack.push(rem);
    div /= 16;
    while (div > 15) {
        rem = div % 16;
        stack.push(rem);
        div /= 16;
    }
    stack.push(div);
    while (!stack.empty()) {
        hex_str += hex(stack.top());
        stack.pop();
    }
    return hex_str;
}
\end{lstlisting}

\subsection{String Padding}
These functions are used to format output fields by adjusting string lengths. 

\begin{itemize}
    \item \texttt{padWithZeros} ensures a string has a specified length by prepending zeros to the front.
    \item \texttt{padWithSpaces} appends spaces to the end to meet the desired length.
\end{itemize}

\begin{lstlisting}[language=C++, caption={padWithZeros() and padWithSpaces()}]
std::string padWithZeros(const std::string& input, int desiredLength) {
    return (input.length() >= desiredLength) ? input :
        std::string(desiredLength - input.length(), '0') + input;
}

std::string padWithSpaces(const std::string& input, int desiredLength) {
    return (input.length() >= desiredLength) ? input :
        input + std::string(desiredLength - input.length(), ' ');
}
\end{lstlisting}

\subsection{Token Splitter}
This function splits a string into tokens based on a specified delimiter. It's especially useful for breaking input lines (like assembly instructions) into label, opcode, and operand fields.

\begin{lstlisting}[language=C++, caption={split() - Tokenizing by delimiter}]
vector<string> split(string str, char del) {
    vector<string> v;
    string temp = "";
    for (int i = 0; i < str.size(); i++) {
        if (str[i] != del) temp += str[i];
        else {
            v.push_back(temp);
            temp = "";
        }
    }
    v.push_back(temp);
    return v;
}
\end{lstlisting}

\subsection{Character to Decimal Conversion}
\textbf{Function:} \texttt{dec()} \\
This function converts a hexadecimal character (0-9, A-F) into its corresponding decimal value. It is used inside the \texttt{hextodec()} function.

\begin{lstlisting}[language=C++, caption={dec() - Hex Character to Decimal}]
int dec(char c) {
    switch (c) {
        case 'A': return 10;
        case 'B': return 11;
        case 'C': return 12;
        case 'D': return 13;
        case 'E': return 14;
        case 'F': return 15;
        default: return stoi(to_string(c)) - 48; // Converts ASCII to int
    }
}
\end{lstlisting}

---

\subsection{Decimal to Hex Character Conversion}
\textbf{Function:} \texttt{hex()} \\
This function is a helper for \texttt{dectohex()} and is used to convert a number between 0-15 into its corresponding hexadecimal character.

\begin{lstlisting}[language=C++, caption={hex() - Decimal (0-15) to Hex Character}]
string hex(int d) {
    switch (d) {
        case 10: return "A";
        case 11: return "B";
        case 12: return "C";
        case 13: return "D";
        case 14: return "E";
        case 15: return "F";
        default: return to_string(d);
    }
}
\end{lstlisting}

---

\subsection{Opcode Table Initialization}
\textbf{Function:} \texttt{const\_optab()} \\
This function creates and returns a map (dictionary) containing mnemonic instructions as keys and their respective opcode values in hexadecimal as values. It is a core part of the assembler’s translation logic.

\begin{lstlisting}[language=C++, caption={const\_optab() - Opcode Table Initialization}]
map<string, string> const_optab() {
    map<string, string> optab;
    optab["ADD"] = "18";
    optab["ADDF"] = "58";
    optab["ADDR"] = "90";
    optab["AND"] = "40";
    optab["CLEAR"] = "B4";
    optab["COMP"] = "28";
    optab["COMPF"] = "88";
    optab["COMPR"] = "A0";
    optab["DIV"] = "24";
    optab["DIVF"] = "64";
    optab["DIVR"] = "9C";
    optab["FIX"] = "C4";
    optab["FLOAT"] = "C0";
    optab["HIO"] = "F4";
    optab["J"] = "3C";
    optab["JEQ"] = "30";
    optab["JGT"] = "34";
    optab["JLT"] = "38";
    optab["JSUB"] = "48";
    optab["LDA"] = "00";
    optab["LDB"] = "68";
    optab["LDCH"] = "50";
    optab["LDF"] = "70";
    optab["LDL"] = "08";
    optab["LDS"] = "6C";
    optab["LDT"] = "74";
    optab["LDX"] = "04";
    optab["LPS"] = "D0";
    optab["MUL"] = "20";
    optab["MULF"] = "60";
    optab["MULR"] = "98";
    optab["NORM"] = "C8";
    optab["OR"] = "44";
    optab["RD"] = "D8";
    optab["RMO"] = "AC";
    optab["RSUB"] = "4C";
    optab["SHIFTL"] = "A4";
    optab["SHIFTR"] = "A8";
    optab["SIO"] = "F0";
    optab["SSK"] = "EC";
    optab["STA"] = "0C";
    optab["STB"] = "78";
    optab["STCH"] = "54";
    optab["STF"] = "80";
    optab["STI"] = "D4";
    optab["STL"] = "14";
    optab["STS"] = "7C";
    optab["STSW"] = "E8";
    optab["STT"] = "84";
    optab["STX"] = "10";
    optab["SUB"] = "1C";
    optab["SUBF"] = "5C";
    optab["SUBR"] = "94";
    optab["SVC"] = "B0";
    optab["TD"] = "E0";
    optab["TIO"] = "F8";
    optab["TIX"] = "2C";
    optab["TIXR"] = "B8";
    optab["WD"] = "DC";
    return optab;
}
\end{lstlisting}

\newpage
\section{Full Code}

\subsection{Operation Table}
% Paste your optab function here
\begin{lstlisting}[language=C++, caption={Operation Table (optab)}]
    #include <iostream>
    #include <string>
    #include <stack>
    #include <map>
    #include <string>
    
    using namespace std;
    
    map <string, string> const_optab()
    {
        map <string, string> optab;
    
            optab["ADD"] = "18";
            optab["ADDF"] = "58";
            optab["ADDR"] = "90";
            optab["AND"] = "40";
            optab["CLEAR"] = "B4";
            optab["COMP"] = "28";
            optab["COMPF"] = "88";
            optab["COMPR"] = "A0";
            optab["DIV"] = "24";
            optab["DIVF"] = "64";
            optab["DIVR"] = "9C";
            optab["FIX"] = "C4";
            optab["FLOAT"] = "C0";
            optab["HIO"] = "F4";
            optab["J"] = "3C";
            optab["JEQ"] = "30";
            optab["JGT"] = "34";
            optab["JLT"] = "38";
            optab["JSUB"] = "48";
            optab["LDA"] = "00";
            optab["LDB"] = "68";
            optab["LDCH"] = "50";
            optab["LDF"] = "70";
            optab["LDL"] = "08";
            optab["LDS"] = "6C";
            optab["LDT"] = "74";
            optab["LDX"] = "04";
            optab["LPS"] = "D0";
            optab["MUL"] = "20";
            optab["MULF"] = "60";
            optab["MULR"] = "98";
            optab["NORM"] = "C8";
            optab["OR"] = "44";
            optab["RD"] = "D8";
            optab["RMO"] = "AC";
            optab["RSUB"] = "4C";
            optab["SHIFTL"] = "A4";
            optab["SHIFTR"] = "A8";
            optab["SIO"] = "F0";
            optab["SSK"] = "EC";
            optab["STA"] = "0C";
            optab["STB"] = "78";
            optab["STCH"] = "54";
            optab["STF"] = "80";
            optab["STI"] = "D4";
            optab["STL"] = "14";
            optab["STS"] = "7C";
            optab["STSW"] = "E8";
            optab["STT"] = "84";
            optab["STX"] = "10";
            optab["SUB"] = "1C";
            optab["SUBF"] = "5C";
            optab["SUBR"] = "94";
            optab["SVC"] = "B0";
            optab["TD"] = "E0";
            optab["TIO"] = "F8";
            optab["TIX"] = "2C";
            optab["TIXR"] = "B8";
            optab["WD"] = "DC";
    
            return optab;
    }
\end{lstlisting}

\subsection{Utility Functions}
% Paste your functions file
\begin{lstlisting}[language=C++, caption={Helper Functions}]
    #include <iostream>
    #include <string>
    #include <stack>
    #include <iomanip>
    #include <sstream>
    using namespace std;
    
    int dec(char c)
    {
        switch (c)
        {
            case 'A':
            return 10;
            case 'B':
            return 11;
            case 'C':
            return 12;
            case 'D':
            return 13;
            case 'E':
            return 14;
            case 'F':
            return 15;
            default:
            return stoi(to_string(c))-48;   //ascii value to magnitude
        }
    }
    
    int hextodec(string hex)
    {
        int len = hex.length();
        int base = 1;
        int dec_num = 0;
        for (int i = len-1; i>=0; i--)
        {
            dec_num += dec(hex[i])*base;
            base = base*16;
        }
        return dec_num;
    }
    
    string hex(int d)
    {
        switch (d)
        {
            case 10:
            return "A";
            case 11:
            return "B";
            case 12:
            return "C";
            case 13:
            return "D";
            case 14:
            return "E";
            case 15:
            return "F";
            default:
            return to_string(d);
        }
    }
    
    string dectohex(int dec_num)
    {
        string hex_str = "";
        stack<int> stack;
    
        int div, rem;
        div = dec_num;
        rem = div % 16;
        //cout << "rem:" << rem << endl;
        stack.push(rem);
        div = (div)/16;
        //cout << "div:" << div << endl;
        while(div > 15)
        {
            rem = div % 16;
            //cout << "rem:" << rem << endl;
            stack.push(rem);
            div = (div)/16;
            //cout << "div:" << div << endl;
        }
        stack.push(div);
        while(!stack.empty())
        {
            //cout << stack.top();
            
            hex_str = hex_str + hex(stack.top());
            stack.pop();
        }
        return hex_str;
    
    }
    
    std::string padWithZeros(const std::string& input, int desiredLength) {
           if (input.length() >= desiredLength) {
            return input;
        } else {
            std::stringstream ss;
            ss << std::string(desiredLength - input.length(), '0') << input;
            return ss.str();
        }
    }
    
    std::string padWithSpaces(const std::string& input, int desiredLength) {
           if (input.length() >= desiredLength) {
            return input;
        } else {
            std::stringstream ss;
            ss << input << std::string(desiredLength - input.length(), ' ');
            return ss.str();
        }
    }
    
    vector<string> split(string str, char del)
    {
          vector<string> v;
          string temp = "";
       
          for(int i=0; i<str.size(); i++)
            {
                if(str[i] != del)
                {
                    temp += str[i];
                }
                else
                {
                    v.push_back(temp);
                    temp = "";
                }
            }
           
          v.push_back(temp);
          return v;
    }
\end{lstlisting}

\subsection{Pass 1}
% Paste your pass1 implementation
\begin{lstlisting}[language=C++, caption={Pass 1 - Intermediate File Generator}]
    #include <iostream>
    #include <fstream>
    #include <string>
    #include <vector>
    #include <map>
    
    #include "Op_tab.h"
    #include "functions.h"
    
    
    using namespace std;
    
    int main()
    {
        int locctr;
        int start_add;
        int prog_len;
        int len;
        string line;
        vector<string> inst_fields;
        map <string, string> OpTab;
        map<string, string> SymTab;
    
        //map <string, string> SymTab;
    
        //Construction of optab
        OpTab = const_optab();
    
        ifstream fin("input.txt");
        ofstream fout("intermediate_file.txt");
        ofstream f2out("SymTab.txt");
    
        getline(fin, line);
        inst_fields = split(line,' ');
        string label = inst_fields[0];
        string opcode = inst_fields[1];
        string operand = inst_fields[2];
        
        if(opcode == "START")
        {
            start_add =  stoi(operand);
            locctr = hextodec(to_string(start_add)); //here locctr is already in hex
            fout << dectohex(locctr) << " " << line << endl;
            getline(fin, line);
            inst_fields = split(line,' ');
            if(inst_fields[0]!="")
            label = inst_fields[0];
            if(inst_fields[1]!="")
            opcode = inst_fields[1];
            if(inst_fields[2]!="")
            operand = inst_fields[2];
        }
        else
        {locctr = 0;}
        
    
        while(opcode != "END")
        {
            if (label != "")
            {
                if(SymTab[label]!="")
                {cout << locctr <<  "Error : Duplicate Symbol " << label << endl;}
                else
                {SymTab[label] = dectohex(locctr);}
            }
    
            fout << dectohex(locctr) << " " << line << endl;
            
            if(OpTab[opcode]!="")
                locctr = locctr + 3;
            else if (opcode == "WORD")
                locctr = locctr + 3;
            else if (opcode == "RESW")
                locctr = locctr + (3*stoi(operand));
            else if (opcode == "RESB")
                locctr = locctr + stoi(operand);
            else if (opcode == "BYTE")
            {
                if(operand[0] == 'C')
                    len = (operand.length() - 3); // removing character {c,' , '}
                else 
                    len = (operand.length() - 3)/2;  //removing the characters {X ' '}
    
                locctr = locctr + len;
            }
            else
                cout << "Error : Invalid operation code" << endl;
            
            //fout << locctr << " " << line << endl;
            label = "";
            getline(fin, line);
            inst_fields = split(line,' ');
            int length = inst_fields.size();
            if(length-- &&  inst_fields[0]!="")
            label = inst_fields[0];
            if(length-- &&  inst_fields[1]!="")
            opcode = inst_fields[1];
            if(length-- &&  inst_fields[2]!="")
            operand = inst_fields[2];
        }
        fout  << " " << line << endl;
    
        prog_len = locctr - start_add + 1;
    
        if (!f2out.is_open()) {
            cerr << "Error: Unable to open SymTab.txt for writing." << endl;
            return 1;
        }
    
        for (const auto& pair : SymTab) {
            f2out << pair.first << " " << pair.second << endl;
        }
    
        fin.close();
        fout.close();
        f2out.close();
        
        return 0;
    }
\end{lstlisting}

\subsection{Pass 2}
% Paste your pass2 implementation
\begin{lstlisting}[language=C++, caption={Pass 2 - Object Code Generator}]
    #include <iostream>
    #include <fstream>
    #include <string>
    #include <vector>
    #include <unordered_map>
    
    #include "Op_tab.h"
    #include "functions.h"
    
    using namespace std;
    
    
    int main()
    {
        string line;
        vector<string> inst_fields;
        map <string, string> OpTab;
        map<string, string> SymTab;
    
        ifstream fin2("SymTab.txt");
    
        if (!fin2.is_open()) {
            cerr << "Error: Unable to open SymTab.txt for reading." << endl;
            return 1;
        }
          
        string line2;
        while (getline(fin2, line2)) {
            // Split each line into key and value
            size_t pos = line2.find(' ');
            if (pos != string::npos) {
                string key = line2.substr(0, pos);
                string value = line2.substr(pos + 1);
                SymTab[key] = value;
            } else {
                cerr << "Error: Invalid line format in SymTab.txt" << endl;
            }
        }
    
        ifstream fin("intermediate_file.txt");
        ofstream fout("obj_prog.txt");
    
        if (!fin.is_open() || !fout.is_open()) {
            cerr << "Error: Unable to open input or output file." << endl;
            return 1;
        }
    
        string locctr;
        string label;
        string opcode;
        string operand;
    
        // Store the position of the first line
        streampos firstLinePos = fin.tellg();
    
        getline(fin, line);
        inst_fields = split(line,' ');
        locctr = inst_fields[0];
        label = inst_fields[1];
        opcode = inst_fields[2];
        operand = inst_fields[3];
    
        int start_add = hextodec(locctr); //here locctr is already in hex
    
        string prevLine;
        string prevAddress;
        while (getline(fin, line)) {
            // Trim leading and trailing whitespace
            line.erase(0, line.find_first_not_of(" \t")); // trim leading whitespace
            line.erase(line.find_last_not_of(" \t") + 1); // trim trailing whitespace
    
            // Store the address from the previous line
            if (!prevLine.empty()) {
                size_t pos = prevLine.find_first_of(" \t");
                prevAddress = prevLine.substr(0, pos);
            }
    
            // Store the current line as the previous line
            prevLine = line;
        }  
    
        // Go back to the first line
        fin.clear(); // Clear any error flags
        fin.seekg(firstLinePos);
        getline(fin, line); // reads first line which is not required
      
        int last_add = hextodec(prevAddress);
    
        int length =  last_add-start_add+1;
        string l = dectohex(length);
        l = padWithZeros(l, 6);
    
        if(opcode == "START")
        {
            fout << "H^" <<  padWithSpaces(label, 6) << "^" << padWithZeros(locctr, 6) << "^" <<  l;
        }
    
    
        OpTab = const_optab();
        getline(fin, line); // reads second line from which we need object code
            
            inst_fields = split(line, ' ');
            locctr = inst_fields[0];
            label = inst_fields[1];
            opcode = inst_fields[2];
            if (inst_fields.size() >= 4) {
            operand = inst_fields[3];
            } else {
            operand = ""; // Set operand to empty string if it's not present
            }
    
        int text_length;
    
        while(opcode != "END"){//text_length > 6){
            text_length = 60;
            fout << '\n' << "T^" << padWithZeros(locctr , 6);
            while (text_length > 0){//opcode != "END") {
                int opAddress;
                string objCode ="";
                if(OpTab.find(opcode)!=OpTab.end())
                {
                    if(operand[operand.length()-1] == 'X' && operand[operand.length()-2] == ',')
                    {
                    int l = hextodec(SymTab[operand.substr(0, operand.length()-2)]);
                    l = l+32768;
                    objCode = OpTab[opcode] + dectohex(l);
                    }
                    else
                    {
                        if(SymTab.find(operand) == SymTab.end())
                            objCode = OpTab[opcode] + "0000";
                        else
                            objCode = OpTab[opcode] + SymTab[operand];
                    }
                    if(text_length >= 6){
                        text_length -= 6;
                        fout << "^" << objCode;
                    }
                    else
                        break;
                }else if(opcode == "BYTE")
                {
                    if(operand[0] == 'C')
                    {
                        for(int i=2; i< operand.length()-1; i++)
                        {
                            char c = operand[i];
                            int asciiValue = c;
                            objCode += (dectohex(asciiValue));
                        }
                    }
                    else 
                    {
                        for(int i=2; i< operand.length()-1; i++)
                        {
                            objCode += operand[i];
                        }
                    }
                    if(objCode.length() < text_length)
                    {
                        text_length -= objCode.length();
                        fout << "^" << objCode;
                    }
                    else{
                        break;
                    }        
                }
                else if(opcode == "WORD")
                {
                    objCode = dectohex(stoi(operand));
                    objCode = padWithZeros(objCode, 6);
                    if(text_length >= 6){
                        text_length -= 6;
                        fout << "^" << objCode;
                    }
                }
                else if(opcode == "RESW" || opcode == "RESB")
                {
                    while(opcode == "RESW" || opcode == "RESB")
                    {
                        if (!getline(fin, line)) {
                        cerr << "Error: Unexpected end of file." << endl;
                        break;
                        }
                        inst_fields = split(line, ' ');
                        locctr = inst_fields[0];
                        label = inst_fields[1];
                        opcode = inst_fields[2];
                        if (inst_fields.size() >= 4) {
                        operand = inst_fields[3];
                        } else {
                        operand = ""; // Set operand to empty string if it's not present
                        }
                    }
                    break;
                }
    
                if (!getline(fin, line)) {
                    cerr << "Error: Unexpected end of file." << endl;
                    break;
                    }
                    inst_fields = split(line, ' ');
                    locctr = inst_fields[0];
                    label = inst_fields[1];
                    opcode = inst_fields[2];
                    if (inst_fields.size() >= 4) {
                    operand = inst_fields[3];
                    } else {
                    operand = ""; // Set operand to empty string if it's not present
                    }
            }
        }
    
        fout << '\n' << "E^" << padWithZeros(dectohex(start_add), 6) << endl;
    
        fin2.close();
        fin.close();
        fout.close();
    
    
    //for adding length of each record
        ifstream fin3("obj_prog.txt");
        ofstream fout2("temp.txt");
        getline(fin3, line); // reads the header record
        fout2 << line << endl;
    
        vector<string> v;
        int text_size;
        while(getline(fin3, line) && line[0] == 'T'){
            v = split(line, '^');
            text_size = 0;
            for(int i=2; i< v.size(); i++)
            {
                text_size += v[i].length();
            }
            line.insert(8, "^"+dectohex(text_size/2));
            fout2 << line << endl;
            v.erase(v.begin());
        }
    
        fout2 << line;
    
        fin3.close();
        fout2.close();
    
        remove("obj_prog.txt");
        rename("temp.txt", "obj_prog.txt");
        return 0;
    }
\end{lstlisting}

\newpage
\section{Input and Output Examples}

\subsection{Input File (input.txt)}
\begin{verbatim}
    COPY START 1000
    FIRST STL RETADR
    CLOOP JSUB RDREC
     LDA LENGTH
     COMP ZERO
     JEQ ENDFIL
     JSUB WRREC
     J CLOOP
    ENDFIL LDA EOF
     STA BUFFER
     LDA THREE
     STA LENGTH
     JSUB WRREC
     LDL RETADR
     RSUB
    EOF BYTE C'EOF'
    THREE WORD 3
    ZERO WORD 0
    RETADR RESW 1
    LENGTH RESW 1
    BUFFER RESB 4096
    RDREC LDX ZERO
     LDA ZERO
    RLOOP TD INPUT
     JEQ RLOOP
     RD INPUT
     COMP ZERO
     JEQ EXIT
     STCH BUFFER,X
     TIX MAXLEN
     JLT RLOOP
    EXIT STX LENGTH
     RSUB
    INPUT BYTE X'F1'
    MAXLEN WORD 4096
    WRREC LDX ZERO
    WLOOP TD OUTPUT
     JEQ WLOOP
     LDCH BUFFER,X
     WD OUTPUT
     TIX LENGTH
     JLT WLOOP
     RSUB
    OUTPUT BYTE X'05'
     END FIRST
\end{verbatim}

\subsection{Intermediate File (intermediate\_file.txt)}
\begin{verbatim}
    1000 COPY START 1000
    1000 FIRST STL RETADR
    1003 CLOOP JSUB RDREC
    1006  LDA LENGTH
    1009  COMP ZERO
    100C  JEQ ENDFIL
    100F  JSUB WRREC
    1012  J CLOOP
    1015 ENDFIL LDA EOF
    1018  STA BUFFER
    101B  LDA THREE
    101E  STA LENGTH
    1021  JSUB WRREC
    1024  LDL RETADR
    1027  RSUB
    102A EOF BYTE C'EOF'
    102D THREE WORD 3
    1030 ZERO WORD 0
    1033 RETADR RESW 1
    1036 LENGTH RESW 1
    1039 BUFFER RESB 4096
    2039 RDREC LDX ZERO
    203C  LDA ZERO
    203F RLOOP TD INPUT
    2042  JEQ RLOOP
    2045  RD INPUT
    2048  COMP ZERO
    204B  JEQ EXIT
    204E  STCH BUFFER,X
    2051  TIX MAXLEN
    2054  JLT RLOOP
    2057 EXIT STX LENGTH
    205A  RSUB
    205D INPUT BYTE X'F1'
    205E MAXLEN WORD 4096
    2061 WRREC LDX ZERO
    2064 WLOOP TD OUTPUT
    2067  JEQ WLOOP
    206A  LDCH BUFFER,X
    206D  WD OUTPUT
    2070  TIX LENGTH
    2073  JLT WLOOP
    2076  RSUB
    2079 OUTPUT BYTE X'05'
      END FIRST
    
\end{verbatim}

\subsection{Symbol Table (SymTab.txt)}
\begin{verbatim}
    BUFFER 1039
    CLOOP 1003
    ENDFIL 1015
    EOF 102A
    EXIT 2057
    FIRST 1000
    INPUT 205D
    LENGTH 1036
    MAXLEN 205E
    OUTPUT 2079
    RDREC 2039
    RETADR 1033
    RLOOP 203F
    THREE 102D
    WLOOP 2064
    WRREC 2061
    ZERO 1030    
\end{verbatim}

\subsection{Object Code (obj\_prog.txt)}
\begin{verbatim}
H^COPY  ^001000^00107A
T^001000^1E^141033^482039^001036^281030^301015^482061^3C1003^00102A^0C1039^00102D
T^00101E^15^0C1036^482061^081033^4C0000^454F46^000003^000000
T^002039^1E^041030^001030^E0205D^30203F^D8205D^281030^302057^549039^2C205E^38203F
T^002057^1C^101036^4C0000^F1^001000^041030^E02079^302064^509039^DC2079^2C1036
T^002073^07^382064^4C0000^05
E^001000
\end{verbatim}

\end{document}