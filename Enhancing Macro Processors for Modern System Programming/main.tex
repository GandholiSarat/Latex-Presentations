\documentclass{beamer}
\usepackage[utf8]{inputenc}
\usepackage{graphicx}
\usepackage{xcolor}
\usepackage{tikz}

\mode<presentation>

%\newif\ifbeamer@secheader
%\beamer@secheaderfalse

%\DeclareOptionBeamer{secheader}{\beamer@secheadertrue}
\ProcessOptionsBeamer

\setbeamertemplate{navigation symbols}{} % Removes navigation buttons


\useoutertheme[footline=authorinstitutetitle,subsection=false]{smoothbars}
\makeatletter
% Custom footline setup
\newcommand{\frameofframes}{/}
\newcommand{\setframeofframes}[1]{\renewcommand{\frameofframes}{#1}}

\setbeamertemplate{footline} 
{%
    \begin{beamercolorbox}[colsep=1.5pt]{upper separation line foot}
    \end{beamercolorbox}
    \begin{beamercolorbox}[ht=2.5ex,dp=1.125ex,%
      leftskip=.3cm,rightskip=.3cm plus1fil]{author in head/foot}%
      \leavevmode{\usebeamerfont{author in head/foot}\insertshortauthor}%
      \hfill%
      {\usebeamerfont{institute in head/foot}\usebeamercolor[fg]{institute in head/foot}\insertshortinstitute}%
    \end{beamercolorbox}%
    \begin{beamercolorbox}[ht=2.5ex,dp=1.125ex,%
      leftskip=.3cm,rightskip=.3cm plus1fil]{title in head/foot}%
      {\usebeamerfont{title in head/foot}\insertshorttitle}%
      \hfill%
      {\usebeamerfont{frame number}\usebeamercolor[fg]{frame number}\insertframenumber~\frameofframes~\inserttotalframenumber}
    \end{beamercolorbox}%
    \begin{beamercolorbox}[colsep=1.5pt]{lower separation line foot}
    \end{beamercolorbox}
    %\hfill \includegraphics[width=0.1\linewidth]{logo.jpg} % Add this line to include the logo
}
\makeatother

% Theme settings
\useinnertheme{circles}

% Color customizations
\xdefinecolor{shisu}{rgb}{0,0.384,0.675}  % RGB #82318E
\setbeamercolor{footline}{bg=shisu}
\setbeamercolor{frametitle}{bg=shisu,fg=white}
\setbeamercolor{title}{bg=shisu}
\setbeamerfont{frametitle}{size=\large}

% Bibliography and caption settings
\setbeamertemplate{bibliography item}[text]
\setbeamertemplate{caption}[numbered]

% Color palettes
\setbeamercolor{palette primary}{use=structure,fg=white,bg=structure.fg}
\setbeamercolor{palette secondary}{use=structure,fg=white,bg=structure.fg!75!black}
\setbeamercolor{palette tertiary}{use=structure,fg=white,bg=structure.fg!50!black}
\setbeamercolor{palette quaternary}{fg=white,bg=structure.fg!50!black}
\setbeamercolor{titlelike}{parent=palette primary}

% Block and sidebar colors
\setbeamercolor{block title}{bg=shisu,fg=white}
\setbeamercolor*{block title example}{use={normal text,example text},bg=white,fg=shisu}
\setbeamercolor{item projected}{fg=white}
\setbeamercolor{sidebar}{bg=shisu}
\setbeamercolor{structure}{fg=shisu}

% Section and subsection styling
\setbeamercolor{section in sidebar}{fg=brown}
\setbeamercolor{subsection in sidebar}{fg=brown}

%\iffalse
% Table of contents at the start of sections/subsections
\AtBeginSection[]{
	\begin{frame}
		\tableofcontents[sectionstyle=show/shaded,subsectionstyle=show/shaded/hide,subsubsectionstyle=show/shaded/hide]
	\end{frame}
}
\AtBeginSubsection[]{
	\begin{frame}
		\tableofcontents[sectionstyle=show/shaded,subsectionstyle=show/shaded/hide,subsubsectionstyle=show/shaded/hide]
	\end{frame}
} 
%\fi

% Add logo overlay in the bottom-right corner
\addtobeamertemplate{background}{%
    \begin{tikzpicture}[remember picture,overlay]
        \node[anchor=south east,xshift= 4pt,yshift=15pt] at (current page.south east) {\includegraphics[width=0.1\linewidth]{logo.jpg}};
    \end{tikzpicture}
}

\title{Enhancing Macro Processors for Modern System Programming}
\author{Gandholi Sarat}
\date{\today}

\begin{document}

\frame{\titlepage}

% New Slide: Research Problem Overview
\begin{frame}{Problems Overview}
    \textbf{1. Optimizing Two-Pass Assemblers for Performance and Memory Efficiency}
    \begin{itemize}
        \item Traditional two-pass assemblers scan code multiple times, increasing memory usage and processing time.
        \item Optimization techniques focus on reducing disk I/O and improving symbol resolution speed.
    \end{itemize}
    \vspace{0.3cm}
    \textbf{2. Improving Linker and Loader Strategies for Dynamic Libraries}
    \begin{itemize}
        \item Dynamic linking reduces binary size but introduces runtime overhead.
        \item Research focuses on caching, prefetching, and improving security (e.g., DLL hijacking prevention).
    \end{itemize}
    \vspace{0.3cm}
    \textbf{3. Enhancing Macro Processors for Modern System Programming}
    \begin{itemize}
        \item ✅ \textbf{Selected this }
    \end{itemize}
\end{frame}

% Slide 1: Introduction
\begin{frame}{Introduction}
    \textbf{What is a Macro Processor?}
    \begin{itemize}
        \item Automates code substitution during preprocessing.
        \item Enables reusable and concise code.
        \item Improves readability and maintainability.
    \end{itemize}
    \textbf{Why are Macro Processors Important?}
    \begin{itemize}
        \item Used in assemblers, compilers, and scripting languages.
        \item Forms the basis of C preprocessor (\#define macros).
        \item Essential for low-level systems programming.
    \end{itemize}
\end{frame}

% Slide 2: Problem Definition
\begin{frame}{Problem Definition}
    \textbf{Challenges in Traditional Macro Processors}
    \begin{itemize}
        \item Limited language support.
        \item Complex build systems.
        \item Lack of debugging support.
        \item Performance overhead.
        \item Limited extensibility.
    \end{itemize}
    \textbf{Research Question:}
    \begin{block}{}
        "How can macro processors be improved to support modern programming paradigms while maintaining efficiency and flexibility?"
    \end{block}
\end{frame}

% Slide 3: Solution Overview
\begin{frame}{Proposed Solutions}
    \begin{itemize}
        \item \textbf{Solution 1:} Language-agnostic macro processor.
        \item \textbf{Solution 2:} Optimized performance with caching and parallel processing.
    \end{itemize}
\end{frame}

% Slide 4: Solution 1 - Language-Agnostic Macro Processor
\begin{frame}{Solution 1: Language-Agnostic Macro Processor}
    \textbf{What is it?}
    \begin{itemize}
        \item A macro processor that can work across multiple programming languages.
        \item Uses Abstract Syntax Trees (ASTs) instead of simple text substitution.
    \end{itemize}
    \textbf{Advantages}
    \begin{itemize}
        \item Improves flexibility and portability.
        \item Reduces syntax errors compared to basic text-based macro expansion.
    \end{itemize}
\end{frame}

% Slide 6: Solution 3 - Performance Optimization
\begin{frame}{Solution 2: Optimized Performance with Caching}
    \textbf{What is it?}
    \begin{itemize}
        \item Uses caching and parallel processing to improve macro expansion speed.
        \item Reduces redundant computations during preprocessing.
    \end{itemize}
    \textbf{Advantages}
    \begin{itemize}
        \item Decreases compilation time for large codebases.
        \item Optimizes CPU and memory usage.
    \end{itemize}
    \textbf{Implementation}
    \begin{itemize}
        \item Use memoization to store frequently expanded macros.
        \item Implement multi-threaded macro expansion.
        \item Leverage GPUs for large-scale macro processing.
    \end{itemize}
\end{frame}
% Slide 5: Overview of Compile-Time Function Memoization
\begin{frame}{Compile-Time Function Memoization Overview}
    \textbf{Core Idea:}
    \begin{itemize}
        \item Memoization saves results of function executions to avoid redundant computations.
        \item This paper proposes applying memoization at compile-time instead of runtime.
    \end{itemize}
    \vspace{0.3cm}
    \textbf{What’s New?}
    \begin{itemize}
        \item Identifies memoizable functions during compilation.
        \item Generates optimized memoization wrappers integrated into the compiled code.
    \end{itemize}
\end{frame}

% Slide 6: Key Features of Compile-Time Function Memoization
\begin{frame}{Key Features of Compile-Time Function Memoization}
    \begin{itemize}
        \item \textbf{Broader Applicability:} Works for all types of functions, including user-defined ones.
        \item \textbf{Inlining for Efficiency:} Memoization wrappers are designed to be inlined, reducing overhead.
        \item \textbf{Handles Complex Scenarios:} Works with global variables, pointers, and constants.
        \item \textbf{Automatic Identification:} Suitable functions are detected automatically during compilation.
    \end{itemize}
\end{frame}

% Slide 7: Implementation of Compile-Time Memoization in LLVM
\begin{frame}{Implementation in LLVM}
    \textbf{How is it implemented?}
    \begin{itemize}
        \item Memoization is added as an optimization pass in the LLVM framework.
        \item \textbf{Step 1:} Analyze functions to identify memoizable candidates.
        \item \textbf{Step 2:} Generate memoization wrappers for each identified function.
        \item \textbf{Step 3:} Replace function calls with calls to memoization wrappers.
    \end{itemize}
    \vspace{0.3cm}
    \textbf{Memoization Wrapper:}
    \begin{itemize}
        \item Checks a table for previously computed results.
        \item If not found, computes the result and stores it.
    \end{itemize}
\end{frame}

% Slide 8: Benefits of Compile-Time Function Memoization
\begin{frame}{Benefits of Compile-Time Function Memoization}
    \begin{itemize}
        \item \textbf{Performance Gains:} Avoids redundant computations and speeds up execution.
        \item \textbf{Reduced Overhead:} Inlining minimizes the cost of memoization.
        \item \textbf{Hardware Extension Proposal:} Suggests optional hardware support for further performance improvements.
    \end{itemize}
\end{frame}


\end{document}

